\title{\bft Программный комплекс автономного подводного аппарата "Акара"}
\author{М.\,Н.\,Чемоданов\footnotemark[1],~
	С.\,С.\,Соавтор\footnotemark[2]\\[2mm]
	\footnotemark[1]~{Студенческое контрукторское бюро СКБ, chemodanov@smtu.ru}\\
	\footnotemark[2]~{Место работы соавтора, coauthor@abcde.ru}~~~}

\begin{document}
\maketitle

\begin{abstract}
	Статья посвящена программному комплексу разрабатываемому в студенческом бюро СПбГМТУ (СКБ СПбГМТУ) автономного подводного аппарата (АНПА) "Акара". В статье рассмотрена разработках всех аспектов программного комплекса - его условно береговая часть (ПО для планирования миссий, ПО для обеспечения связи и позиционирования по гидроакустическому каналу, отладочное ПО) высокоуровневая часть непосредственно подводного аппарата (система управления), низкоуровневая часть (управляющее ПО модулей, система обработки гидроакустических сигналов). Разработка программного обеспечения осуществляется, в основном, силами студентов СПбГМТУ. 
\end{abstract}

%%%%%%%%%%%%%%%%%%%%%%%%%%%%%%%%%%%%%%%%%%%%%%%%%%%%%%%%%%%%%%%%%%%%%%%%%%%%%
\section{Введение}


В настоящее время Студенческим конструкторским бюро (СКБ) Санкт-Петербургского государственного морского технического университета (СПбГМТУ) ведется разработка автономного необитаемого подводного аппарата (АНПА) легкого класса, предназначенного для участия в соревнованиях по морской робототехнике, которые будут проводиться в сентябре 2018 года во Владивостоке\cite{AuvCompetition}.

Задания, поставленные в конкурсе являются упрощенными моделями реальных задач АНПА.

\textbf{Кречин: Краткое описание заданий, выполняемых аппаратом}

\img{general}{Общий вид аппарата}
\textbf{Шестаков: Краткое описание аппарата - размеры, общий вид}

Данная статья посвящена описание разработанного программного комплекса аппарата, во всех его аспектах, выбранному подходу и инструментарию.


	
\section{Описание программного комплекса}
\subsection{Структура программного комплекса}
	\imgscaled{softwarestruct}{Структура программного комплекса}{1}

\subsection{ПО настройки АНПА, планирования миссии и анализа ее выполнения}
	\textbf{Шестаков: на основе ранее написанной статьи}

\subsection{Симулятор выполнения миссии}
	\textbf{Кречин: твой опыт работы с Gazebo}
	
\subsection{ПО высокого уровня}
	\textbf{Кречин: твое видение реализации задач}
	
\subsection{ПО нижнего уровня}
\subsubsection{ПО межмодульного взаимодействия}
	\textbf{Шестаков: почему сделали так, почему модбас, основные принципы, про микроконтроллер}

\subsubsection{ПО модуля гидроакустики}
	\textbf{Базанов, Смирнов: суть задачи, математика задачи, на чем и как реализовывать задачу и т.д.}


\subsubsection{ПО управления движением аппарата}
	\textbf{Кречин: как двигаться по линии, как наводиться на гидроакустический маяк, как попадать в кольцо}

	
\section{Результаты}

Автор - Чемоданов.

\section{Заключение}

Автор - Чемоданов.
