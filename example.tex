\documentclass{article}
\usepackage[utf8]{inputenc}
\usepackage[russian]{babel}

%%% Size definition (do not edit!)
\textwidth=115mm % ???
\textheight=40\baselineskip % 170mm
\footskip=5mm

\setcounter{page}{1}

\newtheorem{lemma}{Лемма}[section]
\newtheorem{remark}{Примечание}[section]
\newtheorem{theorem}{Теоpема}[section]
\newtheorem{statement}{Утвеpждение}[section]
\newtheorem{definition}{Опpеделение}[section]

\def\proof{\par\noindent{\bf Доказательство}. \ignorespaces}
\def\endproof{{\ \vbox{\hrule\hbox{%
   \vrule height1.3ex\hskip1.3ex\vrule}\hrule
  }}\vspace{2mm}\par}

\newcommand{\bft}{\mbox{}\vspace*{-15mm}\mbox{}\\{}\bf{}}
\date{}

%%%%%%%%%%%%%%%%%%%%%%%%%%%%%%%%%%%%%%%%%%%%%%%%%%%%%%%%%%%%%%%%%%%%%%%%%%%%%

\title{\bft Образец оформления статьи для конференции}
\author{А.\,А.\,Автор\footnotemark[1],~
        С.\,С.\,Соавтор\footnotemark[2]\\[2mm]
\footnotemark[1]~{Вычислительный центр РАН, author@ccas.ru}\\
\footnotemark[2]~{Место работы соавтора, coauthor@abcde.ru}~~~}

\begin{document}
\maketitle

\begin{abstract}
Текст аннотации, содержащий краткое описание работы.
Рекомендуемый размер аннотации от 3-х до 5-и предложений.
Для оформления аннотации следует использовать стандартное окружение
\verb"\begin{abstract}...\end{abstract}".
\end{abstract}

%%%%%%%%%%%%%%%%%%%%%%%%%%%%%%%%%%%%%%%%%%%%%%%%%%%%%%%%%%%%%%%%%%%%%%%%%%%%%
\section*{Введение}

Данный файл оформлен в соответствии со всеми требованиями, предъявляемыми к
подготовке публикации на Всероссийской конференции ``Прикладная геометрия,
построение расчетных сеток и высокопроизводительные вычисления'' 2004.
Используйте этот файл в качестве образца оформления статьи.

Название статьи задается с использованием стандартной каманды \verb"\title"
и новой команды \verb"\bft" определенной в начале этого файла,
например, \verb"\title{\bft Название статьи}".
Заглавные буквы в названии следует использовать только в начале названия
и для выделения имен собственных.

Имена авторов задаются с помощью команды \verb"\author", место работы
указывается с помощью специальных символов, задаваемых как 
\verb"\footnotemark[1]", \verb"\footnotemark[2]", \verb"\footnotemark[3]".
Пример использования этих символов смотрите в тексте данного файла.
Информация заданная с помощью комманд \verb"\title" и \verb"\author"
выводится с помощью команды \verb"\maketitle", которая набирается
непосредственно после \verb"\begin{document}".

Некоторые разделы, как например это Введение,
может не иметь номера раздела.
Оформляется это с использованием стандартной комманды
\verb"\section*{Введение}".

Обычный нумерованный раздел оформляется с помощью комманды 
\verb"\section{Название раздела}".

Размер основного шрифта должен быть равен 10 пунктов, с полем текста 
115\,мм на 170\,мм, включая номера страниц.

Статья должна быть набрана используя максимально простое подмножество
\LaTeXe. Для руководства по \LaTeXe\ и в качестве примера оформления ссылок, 
см., например,
\cite{KopkaDaly,KoChe,Lamport,Lvov}
(допускается также написание [1-4]).

%%%%%%%%%%%%%%%%%%%%%%%%%%%%%%%%%%%%%%%%%%%%%%%%%%%%%%%%%%%%%%%%%%%%%%%%%%%%%
\section{Первый нумерованный раздел}

Текст первого нумерованного раздела. Первый абзац каждого раздела всегда 
набирается без отступа.

Пример оформления определения, с использованием определенного 
в начале файла окружения 
\verb"\begin{definition}...\end{definition}".

\begin{definition}
\label{d1}
Текст определения.
\end{definition}

Пример оформления теоремы, с использованием определенного
в начале файла окружения
\verb"\begin{theorem}...\end{theorem}".

\begin{theorem}
\label{t1}
Текст самой теоремы и выносная формула внутри теоремы:
$$
{\rm e}^{2\pi {\rm i}} = 1.
$$
\end{theorem}

\begin{proof}
Текст доказательства Теоремы~\ref{t1}, оформленный с использованием окружения
\verb"\begin{proof}...\end{proof}".
\end{proof}

Пример оформления примечания, с использованием определенного
в начале файла окружения
\verb"\begin{remark}...\end{remark}".

\begin{remark}
\label{rem1}
Текст примечания к Теореме~\ref{t1}
и пример использования автоматической ссылки \verb"\ref{t1}"
на теорему, помеченную как \verb"\label{t1}".
\end{remark}

%%%%%%%%%%%%%%%%%%%%%%%%%%%%%%%%%%%%%%%%%%%%%%%%%%%%%%%%%%%%%%%%%%%%%%%%%%%%%
\section{Название второго раздела}

Текст второго нумерованного раздела.

\subsection{Название подраздела}

Текст первого нумерованного подраздела.

%%%%%%%%%%%%%%%%%%%%%%%%%%%%%%%%%%%%%%%%%%%%%%%%%%%%%%%%%%%%%%%%%%%%%%%%%%%%%
\begin{thebibliography}{9}

\bibitem{KopkaDaly}
  H.\,Kopka and P.\,W.\,Daly,
  \textit{A Guide to \LaTeXe: Document Preparation for Beginners
  and Advanced Users.} Addison-Wesley, 1995.

\bibitem{KoChe}
  И.\,Котельников, П.\,Чеботарев,
  \textit{Издательская система \LaTeXe.}
  Сибирский хронограф, Новосибирск, 1998.

\bibitem{Lamport}
  L.\,Lamport,
  \textit{\LaTeX: A Document Preparation System.}
  Addison-Wesley, 1994.

\bibitem{Lvov}
  С.\,М.\,Львовский,
  \textit{Набор и верстка в пакете \LaTeX.}
  Космосинформ, Москва, 1994.

\end{thebibliography}
\end{document}

